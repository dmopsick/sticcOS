%%%%%%%%%%%%%%%%%%%%%%%%%%%%%%%%%%%%%%%%%
%
% CMPT 424
% Fall 2019
% Lab Two
%
%%%%%%%%%%%%%%%%%%%%%%%%%%%%%%%%%%%%%%%%%

%%%%%%%%%%%%%%%%%%%%%%%%%%%%%%%%%%%%%%%%%
% Short Sectioned Assignment
% LaTeX Template
% Version 1.0 (5/5/12)
%
% This template has been downloaded from: http://www.LaTeXTemplates.com
% Original author: % Frits Wenneker (http://www.howtotex.com)
% License: CC BY-NC-SA 3.0 (http://creativecommons.org/licenses/by-nc-sa/3.0/)
% Modified by Alan G. Labouseur  - alan@labouseur.com
%
%%%%%%%%%%%%%%%%%%%%%%%%%%%%%%%%%%%%%%%%%

%----------------------------------------------------------------------------------------
%	PACKAGES AND OTHER DOCUMENT CONFIGURATIONS
%----------------------------------------------------------------------------------------

\documentclass[letterpaper, 10pt,DIV=13]{scrartcl} 

\usepackage[T1]{fontenc} % Use 8-bit encoding that has 256 glyphs
\usepackage[english]{babel} % English language/hyphenation
\usepackage{amsmath,amsfonts,amsthm,xfrac} % Math packages
\usepackage{sectsty} % Allows customizing section commands
\usepackage{graphicx}
\usepackage[lined,linesnumbered,commentsnumbered]{algorithm2e}
\usepackage{listings}
\usepackage{parskip}
\usepackage{lastpage}

\allsectionsfont{\normalfont\scshape} % Make all section titles in default font and small caps.

\usepackage{fancyhdr} % Custom headers and footers
\pagestyle{fancyplain} % Makes all pages in the document conform to the custom headers and footers

\fancyhead{} % No page header - if you want one, create it in the same way as the footers below
\fancyfoot[L]{} % Empty left footer
\fancyfoot[C]{} % Empty center footer
\fancyfoot[R]{page \thepage\ of \pageref{LastPage}} % Page numbering for right footer

\renewcommand{\headrulewidth}{0pt} % Remove header underlines
\renewcommand{\footrulewidth}{0pt} % Remove footer underlines
\setlength{\headheight}{13.6pt} % Customize the height of the header

\numberwithin{equation}{section} % Number equations within sections (i.e. 1.1, 1.2, 2.1, 2.2 instead of 1, 2, 3, 4)
\numberwithin{figure}{section} % Number figures within sections (i.e. 1.1, 1.2, 2.1, 2.2 instead of 1, 2, 3, 4)
\numberwithin{table}{section} % Number tables within sections (i.e. 1.1, 1.2, 2.1, 2.2 instead of 1, 2, 3, 4)

\setlength\parindent{0pt} % Removes all indentation from paragraphs.

\binoppenalty=3000
\relpenalty=3000

%----------------------------------------------------------------------------------------
%	TITLE SECTION
%----------------------------------------------------------------------------------------

\newcommand{\horrule}[1]{\rule{\linewidth}{#1}} % Create horizontal rule command with 1 argument of height

\title{	
   \normalfont \normalsize 
   \textsc{CMPT 424 - Fall 2019 - Dan Mopsick} \\[10pt] % Header stuff.
   \horrule{0.5pt} \\[0.25cm] 	% Top horizontal rule
   \huge Lab Two  \\     	    % Assignment title
   \horrule{0.5pt} \\[0.25cm] 	% Bottom horizontal rule
}

\author{Dan Mopsick \\ \normalsize Daniel.Mopsick1@Marist.edu}

\date{\normalsize\today} 	% Today's date.

\begin{document}
\maketitle % Print the title

%----------------------------------------------------------------------------------------
%   start PROBLEM ONE
%----------------------------------------------------------------------------------------
\section{Problem One}

\subsection{How	is	your console	like the ancient	TTY	subsystem	in	Unix	as	described	in	
https://www.linusakesson.net/programming/tty/?}

My console is similar to the ancient TTY subsystem in its very basic functions that present the user with very rudimentary editting options of backspace, clearing the screen, and resetting the terminal. The TTY subsystem's buffer also functioned in the same was as my operating system does. We will also be implementing features similar to its session management in later projects. The simplicity of our Operating System is similiar to the TTY subsystem which could only handle simple input and processes.



%----------------------------------------------------------------------------------------
%   start PROBLEM TWO
%----------------------------------------------------------------------------------------
\section{Problem Two}

\subsection{LaTeX?}

Our operating system is not similar to LaTex and neither is TTY. LaTex is a documentation system used for math and computer science. It supports mathematical function documentation and the displaying of code snippets. It functions similar to markup. It is also supports some programmatic functionality such as variables and comments that are invisible in the displayed output. It is powerful than traditional word processors in doing computer science and math work.  



\end{document}
